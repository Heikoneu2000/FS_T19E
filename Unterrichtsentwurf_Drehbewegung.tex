\documentclass[10pt,paper=A4,paper=landscape,headinclude=true,DIV=20]{scrartcl}
%\usepackage[a4paper, landscape, left=2.5cm, right=2.5cm, top=2.5cm, bottom=2.5cm]{geometry}
\usepackage[ngerman]{babel}
%\usepackage{arev}
\usepackage[T1]{fontenc}
\usepackage[scaled]{uarial}
\renewcommand{\familydefault}{\sfdefault}
\usepackage{color}
\usepackage[utf8]{inputenc}
\usepackage{amsmath}
\usepackage{amsfonts}
\usepackage{amssymb}
\usepackage{siunitx}
\usepackage[headsepline]{scrlayer-scrpage}
\usepackage{graphicx}
\pagestyle{scrheadings}
\usepackage{xltabular}
\sisetup{locale = DE}
\usepackage{multirow}
%\usepackage{lscape}
% \usepackage{longtable}
\newcommand{\phyphox}{\textit{phyphox} }
\DeclareUnicodeCharacter{226}{\tests}
\newcommand{\zeit}[2]{#1 \newline \newline #2'}
\begin{document}
\begin{flushright}
Name: Heiko Schröter\\
\end{flushright}
\vspace{0em}
\begin{tabularx}{1\textwidth}{|X|X|X|}
\hline
\multicolumn{3}{|>{\bfseries\Large}X|}{ \vspace{0em}Unterrichtsentwurf}                                                                                                                                                                                                                                                                                                                                                                            \\                                                        
\hline
Richard-Hartmann-Schule & LF/Fach: & Thema: \\
Berufliches Schulzentrum &  & FS Lernfeld 05 Physik T19E \\
für Technik III, Chemnitz & Angewandte Physik & Kräfte bei Drehbewegung \\
\hline 
Klasse: & Planung einer Unterrichtseinheit:  & Datum:  \\ 
&&\\
FS T19E & 90 min & 03. März 2021\\
\hline 
\end{tabularx}\label{tab}
\\\\ \vspace{1em}
\textbf{Ziel: Warum werden die Räder von Fahrzeugen ausgewuchtet?}
\begin{itemize}
\item Was versteht man unter Zentripedalkraft und Zentrifugalkraft?
\item Welche Größen bestimmen die Zentrifugalkraft?
\item Wie lässt sich der Zusammenhang zwischen Zentripedalbeschleunigung $a_z$ und Winkelgeschwindigkeit $\omega$ experimentell bestimmen?
\item Was sind technische Anwendungen von Drehbewegung?
\item Berechnung von Übungsaufgaben
\end{itemize}
% eingefühte Tabelle
\begin{xltabular}{1\textwidth}{|p{2.4em}|p{14cm}|p{6cm}|X|}
\hline
Zeit & Inhalt & Methodisch-didaktisches Vorgehen & {Notizen/ \newline Bemerkungen} \\
\hline\endhead
\zeit{10:45}{10}&\textbf{Stundeneröffnung:} Begrüßung und Reflexion letzte Unterrichtseinheit und Übungsaufgaben, Ergänzung: Grenzen des Hook'schen Gesetzes & Begrüßung durch die Lehrkraft &\\
\hline
\zeit{10:55}{10}&Einführung in das Thema $\rightarrow$ Drehbewegung, Ziele & LV $\rightarrow$ Beamer &\\
\hline
\zeit{11:05}{5}& Kräftegleichgewicht, actio = rectio, Scheinkraft & LV $\rightarrow$ Beamer & \\
\hline
\zeit{11:10}{5}& Theoretische Grundlagen Zentrifugal- und Zentripetalbeschleunigung & LV $\rightarrow$ Beamer & \\
\hline
\zeit{11:15}{5}& Beispiel Karussell & LV $\rightarrow$ Beamer &\\
\hline
\zeit{11:20}{10}& Experimentell Bestimmung des Zusammenhanges zwischen Zentripedalbeschleunigung $a_z$ und Winkelgeschwindigkeit $\omega$ & Demoexperiment & Drehvorrichtung, Handy mit Phyphox, Gnuplot \\
\hline
\zeit{11:30}{10}& Technische Anwendungen von Drehbewegung (Rüttlelplatte, Zentriefuge, Auswuchten eines PKW-Rades & Gruppenarbeit Internetrecherche & Gruppenräume mit 2-3 Schülern\\
\hline
\zeit{11:40}{10}& Übungsaufgaben & LV + EA &\\
\hline
\zeit{12:10}{5}& Fragen/ Wiederholung/ Feedback & &\\
\hline
\zeit{12:15}{0}& Pause & &\\
\hline
\end{xltabular}

\end{document}