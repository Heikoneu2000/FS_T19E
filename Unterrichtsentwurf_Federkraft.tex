\documentclass[10pt,paper=A4,paper=landscape,headinclude=true,DIV=20]{scrartcl}
%\usepackage[a4paper, landscape, left=2.5cm, right=2.5cm, top=2.5cm, bottom=2.5cm]{geometry}
\usepackage[ngerman]{babel}
%\usepackage{arev}
\usepackage[T1]{fontenc}
\usepackage[scaled]{uarial}
\renewcommand{\familydefault}{\sfdefault}
\usepackage{color}
\usepackage[utf8]{inputenc}
\usepackage{amsmath}
\usepackage{amsfonts}
\usepackage{amssymb}
\usepackage{siunitx}
\usepackage[headsepline]{scrlayer-scrpage}
\usepackage{graphicx}
\pagestyle{scrheadings}
\usepackage{xltabular}
\sisetup{locale = DE}
\usepackage{multirow}
%\usepackage{lscape}
% \usepackage{longtable}
\newcommand{\phyphox}{\textit{phyphox} }
\DeclareUnicodeCharacter{226}{\tests}
\newcommand{\zeit}[2]{#1 \newline \newline #2'}
\begin{document}
\begin{flushright}
Name: Heiko Schröter\\
\end{flushright}
\vspace{0em}
\begin{tabularx}{1\textwidth}{|X|X|X|}
\hline
\multicolumn{3}{|>{\bfseries\Large}X|}{ \vspace{0em}Unterrichtsentwurf}                                                                                                                                                                                                                                                                                                                                                                            \\                                                        
\hline
Richard-Hartmann-Schule & LF/Fach: & Thema: \\
Berufliches Schulzentrum &  & FS Lernfeld 5 T19E \\
für Technik III, Chemnitz & Angewandte Physik & Federkraft \\
\hline 
Klasse: & Planung einer Unterrichtseinheit:  & Datum:  \\ 
&&\\
FS T19E & 90 min & 15. Februar 2021\\
\hline 
\end{tabularx}\label{tab}
\\\\ \vspace{1em}
\textbf{Ziel: Kennen und anwenden vom Hook'schen\footnote{Robert Hook, engl. Physiker (1635 bis 1703)}  Gesetz}
\begin{itemize}
\item Wie verändert sich die Länge einer Feder bei Einwirkung einer Kraft?
\item Wie ist ein Federkraftmesser aufgebaut?
\item Welche weiteren Möglichkeiten zur Kraftmessung gibt es?
\item Berechnung der Änderung der Bodenfreiheit beim beladen eines LKW.
\item Wo liegen die Grenzen zur Anwendung vom Hook'schen Gesetz?
\end{itemize}
% eingefühte Tabelle
\begin{xltabular}{1\textwidth}{|p{2.4em}|p{14cm}|p{6cm}|X|}
\hline
Zeit & Inhalt & Methodisch-didaktisches Vorgehen & {Notizen/ \newline Bemerkungen} \\
\hline\endhead
\zeit{10:45}{10}&\textbf{Stundeneröffnung} (Begrüßung) & Begrüßung durch die Lehrkraft (Vorstellung, Erwartungen). Dialog &\\
\hline
\zeit{10:55}{10}&Einführung in das Thema $\rightarrow$ Federkraft, Ziele & LV $\rightarrow$ Beamer &\\
\hline
\zeit{11:05}{5}& Kraft-Verlängerungs-Diagramm & LV $\rightarrow$ Beamer & \\
\hline
\zeit{11:10}{5}& Das Hook'sche Gesetz  & LV $\rightarrow$ Beamer & \\
\hline
\zeit{11:15}{5}& Aufbau eines Federkraftmessers & LV $\rightarrow$ Beamer &\\
\hline
\zeit{11:20}{10}& Weitere Möglichkeiten der Kraftmessung & Gruppenarbeit, Tafel, Beamer & \\
\hline
\zeit{11:30}{10}& Berechnung der Formänderung bei Krafteinwirkung & Tafel, Beamer &\\
\hline
\zeit{11:40}{10}& Übungsaufgabe & EA &\\
\hline
\zeit{11:50}{20}& Spannungs-Dehnungs-Schaubild & LV $\rightarrow$ Beamer & \\
\hline
\zeit{12:10}{5}& Fragen/ Wiederholung/ Feedback & &\\
\hline
\zeit{12:15}{0}& Pause & &\\
\hline
\end{xltabular}

\end{document}