\documentclass{beamer}
\usepackage[ngerman]{babel}
\usepackage[utf8]{inputenc}
\usepackage{amsmath}
\usepackage{amsthm}
\usepackage{siunitx}
\usepackage{graphicx}
\usepackage{pgfplots}
\sisetup{locale = DE}
% Lade Beamer Stile
\usepackage{beamerthemesplit}
\usepackage{tcolorbox}
\usetheme{Rochester}
\usecolortheme{crane}


\title{Unterrichtseinheit zur Federkraft}
\subtitle{Das Hook'sche\footnote{Robert Hook, engl. Physiker (1635 bis 1703)}  Gesetz}
\author{Heiko Schröter}
\date{\today}

\setbeamertemplate{enumerate item}{\alph{enumi})}

\begin{document}

\frame{
\frametitle{Beispielaufgabe 2}
\textbf{Aufgabe:}\\
Die Federkonstante einer Schraubenfeder beträgt $\SI{15}{\frac{\newton}{\meter}}$. Wie groß ist die Verkürzung der Feder, wenn sie mit
\begin{itemize}
\item [a)] $\SI{1.5}{\newton}$
\item [b)] $\SI{5}{\newton}$
\end{itemize}
zusammengedrückt wird?\\
}
\frame{
\frametitle{Beispielaufgabe 3}
\textbf{Aufgabe:}\\
Eine Feder wird durch Anhängen eines Körpers A um $\SI{100}{\milli\meter}$, durch Anhängen eines Körpers B um $\SI{40}{\milli\meter}$ verlängert. Der Körper B zieht mit der Kraft $\SI{2}{\newton}$. Mit welcher Kraft zieht der Körper A an der Feder?\\
}
\end{document}
