\documentclass[12pt,oneside]{scrartcl}
\usepackage[a4paper, left=2.5cm, right=2cm, top=5.5cm, bottom=2.5cm]{geometry}
\usepackage[ngerman]{babel}
\usepackage{color}
\usepackage[utf8]{inputenc}
\usepackage{amsmath}
\usepackage{amsfonts}
\usepackage{amssymb}
\usepackage{siunitx}
\usepackage{graphicx}
\usepackage{pgfplots}
\usepackage{hyperref}
\sisetup{locale = DE}
\newcommand{\phyphox}{\textit{phyphox} }
\usepackage{biblatex}
\usepackage{xltabular}
\renewcommand{\labelenumi}{\alph{enumi})}
\addbibresource{Schallgeschwindigkeit.bib}
\usepackage{scrlayer-scrpage}
\newcolumntype{C}[1]{>{\centering\arraybackslash}p{#1}}
\DeclareNewLayer[%
  background,% Hintergrundebene
  width=\textwidth,% Größe und Position des Textbereichs
  hoffset=2.5cm,
  voffset=2.5cm,
  align=tl,
  contents={%
  \sffamily
    \setlength{\fboxsep}{2mm}% Abstand Box Text
    \setlength{\fboxrule}{.5mm}% Dicke Linie
    % vertikale Position korrigieren    
    \vskip-\fboxsep\vskip-\fboxrule\vskip-\baselineskip
    % horizontale Position korrigieren
    \hspace{-\fboxsep}\hspace{-\fboxrule}%
        
    \begin{tabularx}{\textwidth}{|>{\bfseries\Large}C{7cm}|X|X|}                                                       
\hline
\vspace{0em}\includegraphics[scale=0.3]{logo2.png} & Name: & Datum: \\
\hline

\end{tabularx}
    %\makebox[\layerwidth][l]{%
    %  \fbox{\rule{0pt}{\layerheight}\rule{\layerwidth}{0pt}}% Rahmen
    %}%
  }
]{frame}
\AddLayersToPageStyle{@everystyle@}{frame}

\begin{document}
\sffamily

\title{Mechanik}
\subtitle{Übungsaufgaben zur Arbeit, Leistung, Impuls und Stoß}
\date{}
\maketitle

\section{Aufgabe (Arbeit)}
	Ein Monteur besteigt einen Freileitungsmast und wendet dabei eine Kraft von \SI{750}{\newton} auf. Welche Arbeit vollbringt er bei einer Steighöhe von \SI{12}{\meter}?\\
	\begin{tikzpicture}
		\draw[step=0.5cm,gray,very thin] (0,0) grid (16.5,3);
	\end{tikzpicture}
	
\section{Aufgabe (Arbeit)}
	Ein Lastenaufzug bewegt eine Masse von \SI{3500}{\kilogram} um \SI{18}{\meter} nach unten. Berechnen Sie die vom Aufzug aufgenommene Arbeit ($g\approx\SI{10}{\newton\per\kilogram}$).\\
	\begin{tikzpicture}
		\draw[step=0.5cm,gray,very thin] (0,0) grid (16.5,3);
	\end{tikzpicture}

\section{Aufgabe (Leistung)}
	Ein Güterzug mit 55 Wagen zu je \SI{20}{\tonne} soll in \SI{8}{\minute} eine Steigung mit \SI{225}{\meter} Höhenunterschied hinaufgezogen werden. Reibungswiderstand und Luftwiderstand bleiben unberücksichtigt, $g\approx\SI{10}{\newton\per\kilogram}$. Berechnen Sie:
	\begin{enumerate}
	\item die Arbeit,
	\item die Leistung.
	\end{enumerate}
	\begin{tikzpicture}
		\draw[step=0.5cm,gray,very thin] (0,0) grid (16.5,4);
	\end{tikzpicture}

\section{Aufgabe (Leistung)}
	Ein Wasserkraftwerk hat die Fallhöhe von \SI{12}{\meter} bei einem Wasserstrom von \SI{60}{\cubic\meter\per\second}. Wie groß ist die Leistungsaufnahme der Turbine ($g=\SI{9,81}{\newton\per\kilogram}$)?\\
	\begin{tikzpicture}
		\draw[step=0.5cm,gray,very thin] (0,0) grid (16.5,3);
	\end{tikzpicture}
	
\section{Aufgabe (Kraftstoß)}
	Ein PKW mit der Masse \SI{1400}{\kilogram} und der Geschwindigkeit \SI{36}{\kilo\meter\per\hour} erhält von einem von hinten auffahrenden Wagen einen Kraftstoß \SI{2400}{\newton\second}.
	\begin{enumerate}
	\item Wie groß ist seine Geschwindigkeit unmittelbar nach dem Unfall?
	\item Welchen Kraftstoß erfuhr ein Insasse mit der Masse \SI{70}{\kilogram}?
	\item Welche Welche Kraft wirkte auf das Fahrzeug, wenn es \SI{0,5}{\second} lang beschleunigt worden ist?
	\end{enumerate}
	\begin{tikzpicture}
		\draw[step=0.5cm,gray,very thin] (0,0) grid (16.5,5);
	\end{tikzpicture}
	
\section{Aufgabe (Impuls)}
	Ein Eisenbahnwagen mit der Masse \SI{10}{\tonne} rollt mit einer Geschwindigkeit \SI{1,6}{\meter\per\second} gegen einen Prellbock, von dem er nach \SI{0,4}{\second} gleich schnell zurückprallt. Wie groß sind
	\begin{enumerate}
	\item die Impulsänderung,
	\item der Kraftstoß,
	\item die mittlere Kraft auf den Wagen bzw. auf den Prellbock?
	\end{enumerate}
	\begin{tikzpicture}
		\draw[step=0.5cm,gray,very thin] (0,0) grid (16.5,5);
	\end{tikzpicture}
	
\section{Aufgabe (Impuls)}
	Ein Hammer (Masse \SI{0,2}{\kilogram}) trifft mit der Geschwindigkeit \SI{8}{\meter\per\second} einen Nagel, der \SI{5}{\milli\second} lang in das Holz getrieben wird. Wie groß sind
	\begin{enumerate}
	\item die Impulsänderung des Hammers,
	\item der Kraftstoß auf den Hammer,
	\item die Kraft auf den Nagel?
	\end{enumerate}
	\begin{tikzpicture}
		\draw[step=0.5cm,gray,very thin] (0,0) grid (16.5,5);
	\end{tikzpicture}
		
\end{document}