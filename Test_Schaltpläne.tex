\documentclass[a4paper]{scrartcl}
\usepackage[
    typ=lzk,
    klausurtyp=kurs,
    klasse=scrartcl,
    fach=Physik,
    lerngruppe=T20,
    loesungen=seite,
    module={Symbole,Bewertung} 
    %erwartungshorizontAnzeigen,
    %erwartungshorizontStil=simpel,
hHh]{schule}
% Dieses Dokument gehört zu den Beispiel des LaTeX Paketes Schule und ist von den Autoren
% des Pakets erstellt worden.
%
% Das Dokument steht unter der Lizenz: Creative Commons by-nc-sa Version 4.0
% http://creativecommons.org/licenses/by-nc-sa/4.0/deed.de
%
% Nach dieser Lizenz darf das Dokument beliebig kopiert und bearbeitet werden,
% sofern das Folgeprodukt wiederum unter gleichen Lizenzbedingungen vertrieben
% und auf die ursprünglichen Urheber verwiesen wird.
% Eine kommerzielle Nutzung ist ausdrücklich ausgeschlossen.

\author{Heiko Schröter}
\date{\today}
\title{Wärmelehre}

\usepackage{amsmath}
\usepackage{pgfplots}
\usepackage{mathrsfs}
\usepackage{xfrac}
\usepackage{siunitx}
%\sisetup{quotient-mode = fraction, locale = DE}
\sisetup{per-mode = fraction, locale = DE}
\usepackage{blindtext}
%\usepackage{units}
%\usepackage{circuittikz}
%\usepackage{mhchem}
\newcommand{\Ergebnis}[1]{\underline{\underline{#1}}}
\begin{document}
\punktuebersicht*

\setzeAufgabentemplate{schule-randpunkte}  
\begin{aufgabe}[points={6}]
\begin{circuitikz}
\draw
(0,0)--(1,0) to[european resistor,l=$47$\,k$\Omega$] (3,0)--(5,0)
to[C, l=$470$\,$\mu$F] (7,0) -- (8,0)
(4.5,0) to[short, -*] (4.5,0) -- (4.5, -2)
(4.5,-2) -- (5,-2) to[voltmeter, l=$U_C$] (7,-2) -- (7.5,-2)
(7.5, -2) to[short, -*] (7.5,0)
(8,1) node[spdt, rotate=90] (Ums) {}
(Ums) node[right=0.4cm] {$WS$}
(Ums.out 1) node[left] {1}
(Ums.out 2) node[right] {2}
(0,0) |- (2,4) to[closing switch, l=$S$] (3,4) to[battery1,
l=$U$] (5,4) -| (Ums.out 2)
(Ums.in) -- (8,0)
(Ums.out 1) |- (0,2) to[short, -*] (0,2)
 ;
\end{circuitikz}
\begin{loesung}
Nichts!
\end{loesung}
\end{aufgabe}

\begin{aufgabe}[points={2}]
\begin{circuitikz}[european, voltage shift =0.5]
\draw (0,0) to[isourceC, l=$I_0$, v=$V _0$] (0,3)
to[short, -*, f=$I_0$] (2,3)
to[R=$R_1$, f>_=$i_1$] (2,0) -- (0,0);
\draw (2,3) -- (4,3)
to[R=$R_2$, f>_=$i_2$]
(4,0) to[short, -*] (2,0);
\end{circuitikz}
\begin{loesung}
Nichts!
\end{loesung}
\end{aufgabe}

\end{document}