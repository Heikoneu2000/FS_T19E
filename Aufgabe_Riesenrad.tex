\documentclass{beamer}
\usepackage[ngerman]{babel}
\usepackage[utf8]{inputenc}
\usepackage{setspace} 
\usepackage{amsmath}
\usepackage{amsthm}
\usepackage{pgfplots}
\usepackage{siunitx}
\usepackage{eurosym}
\sisetup{locale = DE}
\sisetup{per-mode=fraction}
% Lade Beamer Stile
\usepackage{beamerthemesplit}
\usetheme{Rochester}
\usecolortheme{crane}
\setbeamercolor{bgcolor}{fg=black,bg=green!15!white}
\title{Übungsaufgabe Riesenrad}
\subtitle{Kräfte bei Drehbewegung}
\author{Heiko Schröter}
\date{\today}

\setbeamertemplate{enumerate item}{\alph{enumi})}
\newtheorem{satz}{Satz}

\begin{document}

\frame{\titlepage}

\frame
{
  \frametitle{Aufgabe}
Ein Riesenrad hat einen Durchmesser von $\SI{12}{\meter}$ und dreht sich in der Minute 4 mal. Das Gewicht einer Kabine beträgt bei voller Besetzung $F_G=\SI{3000}{\newton}$. Berechnen Sie
\begin{enumerate}
\item die Umfangsgeschwindigkeit $v_u$ in \si{\frac{\meter}{\second}} für $d=\SI{12}{\meter}$,
\item die Winkelgeschwindigkeit $\omega$,
\item die auf die Kabine wirkende Fliehkraft $F_Z$, wenn der Kabinenschwerpunkt auf $d=\SI{12}{\meter}$ liegt,
\item die nach unten wirkende Kraft der Kabine, wenn sich diese durch den untersten Punkt des umfahrenen Kreises bewegt.
\end{enumerate}
}

\frame[allowframebreaks]
{
  \frametitle{Lösung}
\begin{enumerate}
\item
	\begin{align*}
	v_u&=\dfrac{d\cdot\pi\cdot n}{60}=\dfrac{\SI{12}{\meter}\cdot\pi\cdot 4}{\SI{60}{\second}}=\SI{2,513}{\frac{\meter}{\second}}\\
	\end{align*}
\item 
	\begin{align*}
	\omega&=\dfrac{\pi\cdot n}{30}=\dfrac{\pi\cdot 4}{30}\si{\radian\per\second}=\SI{0,4189}{\radian\per\second}\\
	\end{align*}
\item 
	\begin{align*}
	F_Z&=m\cdot r\cdot\omega^{2}=\dfrac{F_G}{g}\cdot r\cdot\omega^{2}=\dfrac{\SI{3000}{\newton}}{\SI{9,81}{\meter\per\square\second}}\cdot \SI{6}{\meter}\cdot(\SI{0,4189}{\per\second})^{2}\\
	&=\SI{322}{\newton}\\
	\end{align*}
\item 
	\begin{align*}
	F_r=F_Z+F_G=\SI{322}{\newton}+\SI{3000}{\newton}=\SI{3322}{\newton}
	\end{align*}
\end{enumerate}
}

\end{document}